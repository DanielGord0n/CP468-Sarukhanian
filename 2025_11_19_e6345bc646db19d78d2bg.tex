\documentclass[10pt]{article}
\usepackage[russian]{babel}
\usepackage[utf8]{inputenc}
\usepackage[T2A]{fontenc}
\usepackage{amsmath}
\usepackage{amsfonts}
\usepackage{amssymb}
\usepackage[version=4]{mhchem}
\usepackage{stmaryrd}
\usepackage{fvextra, csquotes}

\title{ЗАМЕТКА О ПОСТРОЕНИИ $\delta$-КОДОВ }

\author{Квашенников В. В., Яковлев В. Г.}
\date{}


\begin{document}
\maketitle
\section*{Саруханяи А. Г.}
\begin{displayquote}
Приводятся конструкции двух новых классов циклических $T$-матриц, что приводит к построению матриц Адамара новых порядков.
\end{displayquote}

\section*{§ 1. Введение}
Один из основных методов построения матриц Адамара, задача существования которых для всех порядков $m$ вида $m \equiv 0(\bmod 4)$ до сих пор не решена, основан на построении массивов Бомера - Холла, которые, в свою очередь, синтезируются с помощью массивов Геталса - Зейделя и $T$-матриц [1-3].

В работе [2] задача построения циклических $T$-матриц сводится к построению четырехсимвольных $\delta$-кодов длины $n$.

Определение 1 [4]. $\left(a_{i}\right)_{i=1}^{n},\left(b_{i}\right)_{i=1}^{n},\left(c_{i}\right)_{i=1}^{n},\left(d_{i}\right)_{i=1}^{n}, a_{i}, b_{i}, c_{i}, d_{i} \in\{-1$, +1 ) называются дополнительными последовательностями длины $n$, если выполняется условие $\sum_{i=j}^{n-j}\left(a_{i} a_{i+j}+b_{i} b_{i+j}+c_{i} c_{i+j}+d_{i} d_{i+j}\right)=0, j=1,2, \ldots$ ..,$n-1$.

Заметим, что пары дополнительных последовательностей называются дополнительными последовательностями Голея $[2,4]$.

В работах [5, 6] с помощью дополнительных последовательностей Голея и Турина [4] были построены $\delta$-коды длины $3 t, 7 t, 13 t$, где $t \in L_{1}= =\left\{3,5, \ldots, 59,2^{a} 10^{b} 26^{c}+1\right\}, a, b, c$ - целые неотрицательные числа.

В настоящей заметке приводятся конструкции лвух новых классов циклических $T$-матриц, что приводит к построению матриц Адамара новых порядков.

\section*{§ 2. Циклические $\boldsymbol{T}$-матрицы порядка $2 \cdot 11(2 n-1), 2(2 n-1)(2 k+1)$}
Из дополнительных последовательностей длины $n$ образуем последовательность векторов $V=\left\{V_{i}=\left(a_{i}, b_{i}, c_{i}, d_{i}\right)\right\}$. Очевидно, что

$$
\sum_{i=1}^{n-j} V_{i} V_{i+j}=0, \quad j=1,2, \ldots, n-1 .
$$

Если последовательность $V$ образована с помощью $t(t \leqslant 4)$ взаимно ортогональных векторов, то $V$ будем называть $\delta(t, n)$-последовательностью [7, 8]. Если $V$ образована произвольными четырехмерными векторами, то $V$ назовем $\delta$-последовательностью.

Теорема 1. Пусть существует $\delta$-последовательность длины п. Тогда существует $\delta(4,2 n)$-последовательность.

Доказательство. Пусть $V=\left\{\left(a_{i}, \quad b_{i}, \quad c_{i}, \quad d_{i}\right)\right\}_{i=1}^{n}-\delta$-последовательность. Докажем, что $P=\left\{\left(a_{i}, a_{i}, c_{i}, c_{i}\right)_{i=1}^{n},\left(b_{i},-b_{i}, d_{i},-d_{i}\right)_{i=1}^{n}\right\}$ является $\delta(4,2 n)$-последовательностью.

Можно показать, что непериодическая автокорреляционная функция последовательности $Q=\left\{\left(x_{i}\right)_{i=1}^{m},\left(y_{i}\right)_{i=1}^{n}\right\}, \quad m \leqslant n$ имеет вид

\[
N_{Q}(j)=\left\{\begin{array}{l}
\sum_{i=1}^{m-j} x_{i} x_{i+j}+\sum_{i=1}^{j} y_{i} x_{m+i-j}+\sum_{i=1}^{n-j} y_{i} y_{i+j}, \quad 1 \leqslant j \leqslant m-1,  \tag{1}\\
\sum_{\substack{i=1 \\
m}}^{m-j} x_{i} y_{i+j-m}+\sum_{i=1}^{m-j} y_{i} y_{i+j}, \quad m \leqslant j \leqslant n-1, \\
\sum_{i=1}^{m-j} x_{i} y_{i+j-m}, \quad n \leqslant j \leqslant m+n-1 .
\end{array}\right.
\]

Теперь, применяя формулы (1) для последовательности $p$, легко доказать, что $N_{P}(j)=0$. Очевидно, что последовательность $P$ образована четырьмя взаимно ортогональными векторами. Теорема доказана.

Утверждение 1. Пусть существуют последовательности Турина длины $n$. Тогда существует $\delta(4,2 \cdot 11(2 n-1))$-последовательность.

Доказательство Пусть $A=\left(a_{i}\right)_{i=1}^{n}, B=\left(b_{i}\right)_{i=1}^{n}, C=\left(c_{i}\right)_{i=1}^{n-1}, D= =\left(d_{i}\right)_{i=1}^{n-1}$ - последовательности Турина, $x=(1,1,1,1), y=(1,1,-1,-1) z=(-1,1,-1,1), w=(-1,1,1,-1)$.

Рассмотрим последовательность векторов $x, y, z, w$ :

$$
\begin{aligned}
& X=\left\{\left(x a_{i}\right)_{i=1}^{n},\left(x c_{i}\right)_{i=1}^{n-1},\left(-x a_{i}\right)_{i=1}^{n},\left(-x c_{i}\right)_{i=1}^{n-1},\left(-x b_{n-i+1}\right)_{i=1}^{n},\right. \\
& \left(-x c_{i}\right)_{i=1}^{n-1},\left(-x a_{i}\right)_{i=1}^{n},\left(x c_{i}\right)_{i=1}^{n-1},\left(y a_{i}\right)_{i=1}^{n},\left(x d_{i}\right){ }_{i=1}^{n-1}, \\
& \left(y a_{i}\right)_{i=i}^{n},\left(x d_{i}\right)_{i=1}^{n-1},\left(y a_{i}\right)_{i=1}^{n},\left(x d_{i}\right)_{i=1}^{n-1},\left(y b_{i}\right)_{i=1}^{n},\left(y d_{i}\right)_{i=1}^{n-1}, \\
& \left(-y b_{i}\right)_{i=1}^{n},\left(y c_{n-i}\right)_{i=1}^{n-1},\left(-y b_{i}\right)_{i=1}^{n},\left(y d_{i}\right)_{i=1}^{n-1},\left(y b_{i}\right)_{i=1}^{n}, \\
& \left(-y d_{i}\right)_{i=1}^{n-1},\left(z a_{i}\right)_{i=1}^{n},\left(z c_{i}\right)_{i=1}^{n-1},\left(-z a_{i}\right)_{i=1}^{n},\left(z d_{n-i}\right)_{i=1}^{n-1},\left(-z a_{i}\right)_{i=1}^{n}, \\
& \left(z c_{i}\right)_{i=1}^{n-1},\left(z a_{i}\right)_{i=1}^{n},\left(-z c_{i}\right)_{i=1}^{n-1},\left(-z b_{i}\right)_{i=1}^{n},\left(-w c_{i}\right)_{i=1}^{n-1}, \\
& \left(-z b_{i}\right)_{i=1}^{n},\left(-w c_{i}\right)_{i=1}^{n-1},\left(-z b_{i}\right)_{i=1}^{n},\left(-w c_{i}\right)_{i=1}^{n-1},\left(w b_{i}\right)_{i=1}^{n}, \\
& \left(w d_{i}\right)_{i=1}^{n-1},\left(-w b_{i}\right)_{i=1}^{n},\left(-w d_{i}\right)_{i=1}^{n-1},\left(w a_{n-i+1}\right)_{i=1}^{n}, \\
& \left.\left(-w d_{i}\right)_{i=1}^{n-1},\left(-w b_{i}\right)_{i=1}^{n},\left(w d_{i}\right)_{i=1}^{n-1}\right\} .
\end{aligned}
$$

Легко доказать, что $X$ является $\delta(4,2 \cdot 11(2 n-1))$-последовательностью.

Утверждение 2. Пусть существуют последовательности Турина и Голея соответственно длины $n$ и $k$. Тогда существует $\delta(4,2(2 n-1) \times \times(2 k+1))$-последовательность.

Доказательство. Пусть $A=\left(a_{i}\right)_{i=1}^{n}, B=\left(b_{i}\right)_{i=1}^{n}, C=\left(c_{i}\right)_{i=1}^{n-1}, D= =\left(d_{i}\right)_{i=1}^{n-1}$ и $F=\left(f_{i}\right)_{i=1}^{i t}, G=\left(g_{i}\right)_{i=1}^{l_{i}}$ - последовательности Турина и Голея, $x=(1,1,0,0), y=(1,-1,0,0), z=(0,0,1,1), w=(0,0,1,-1)$. Синтезируем последовательность

$$
\begin{aligned}
& X=\left\{\left\{\left\{a_{i}\left(x f_{k-j+1}+z g_{k-j+1}\right)\right\}_{i=1}^{n}, \quad\left\{c_{i}\left(x g_{j}+z f_{k-j+1}\right)\right\}_{i=1}^{n-1}\right\}_{j=1}^{k}\right. \\
& \left\{x a_{i}-z b_{i}\right\}_{i=1}^{n}, \quad\left\{x d_{i}-z c_{i}\right\}_{i=1}^{n-1}, \quad\left\{\left\{b_{i}\left(x g_{j}+z f_{k-j+1}\right)\right\}_{i=1}^{n}\right. \\
& \left.\left\{d_{i}\left(-x f_{j}+z g_{k-j+1}\right)\right\}_{i=1}^{n-1}\right\}_{j=1}^{k}, \quad\left\{\left\{a_{i}\left(y f_{k-j+1}+w g_{k-j+1}\right)\right\}_{i=1}^{n}\right. \\
& \left.\left\{c_{i}\left(y g_{j}-w f_{j}\right)\right\}_{i=1}^{n-1}\right\}_{j=1}^{k}, \quad\left\{-y a_{i}+w b_{i}\right\}_{i=1}^{n}, \quad\left\{y d_{i}+w c_{i}\right\}_{i=1}^{n-1}
\end{aligned}
$$

$$
\left.\left\{\left\{-b_{i}\left(y g_{j}+w f_{k-j+1}\right)\right\}_{i=1}^{n}, \quad\left\{d_{i}\left(y f_{j}-w g_{k-j+1}\right)\right\}_{i=1}^{n-1}\right\}_{i=1}^{k}\right\} .
$$

Можно показать, что $X$ является $\delta(4,2(2 n-1)(2 k+1))$-последовательностью.

Следствие 1. Существуют четыре циклические Т-матрицы порядка $2 \cdot 11(2 n-1)$ и $2(2 n-1)(2 k+1)$, где $k=2^{a} 10^{b} 26^{c}, n \in L_{2}=\{2,3,4, \ldots, 8,13$, $\left.15,2^{d} 10^{e} 26^{f}+1\right\}, a, b, \ldots, f$ - целье неотрицательные числа.

Доказательство следствия 1 опирается на утверждения 1,2 , а также на работу [7], где, исходя из $\delta(4, n)$-последовательности, приведена конструкция циклических $T$-матриц.

Ценность следствия 1 заключается в возможности построения циклических $T$-матриц порядка $2 m_{1}$ и $2 m_{2}$, хотя существование циклических $T$-матриц порядка $m_{1}=11(2 n-1)$ и $m_{2}=(2 n-1)(2 k+1)$ неизвестно.

Следствие 2. Существуют матриць Адамара порядка $8 \cdot 11(2 n-1) m$ и $8 \cdot(2 n-1)(2 k+1) m$, где $m$ - порядок существующих матриц типа Вильямсона [1,9]. В частности $m \in\left\{3,5,7, \ldots, 31,33,43,3^{a}(p+1) / 2\right\}$, где а целое положительное число, $p \equiv 1(\bmod 4)$ - степень простого числа.

Доказательство получается из теорем Купера - Валлиса и Бомера Холла [1].

\section*{ЛИТЕРАТУРА}
\begin{enumerate}
  \item Wallis W. D., Street A. P., Wallis J. S. Combinatorics: Room Squares, Sum-Free Sets, Hadamard Matrices. Lecture Notes in Mathematics. Berlin - New York: SpringerVerlag, 1972. V. 292.
  \item Turyn R. J. Hadamard Matrices, Baumert-Hall Units, Four-Symbol Sequences, Puls Compression, and Surface Wave Encodings // J. Comb. Theory. 1974. V. 16(A). № 3. P. 313-333.
  \item Саруханян A. Г. О массивах типа Геталса - Зейделя // Уч. записки Ереван. гос. ун-та. 1979. № 1. С. 12-19.
  \item Robinson P. J., Wallis J. S. A note on using sequences to construct orthogonal designs // Colloquia Math. Soc. Jánis Bolyai. Combinatorics. Budapest: Akad. Kiado. 1976. V. 18. P. 911-932.
  \item Yang C. H. Hadamard Matrices and $\delta$-Codes of Length $3 n / /$ Proc. Amer. Math. Soc. 1982. V. 85. № 3. P. 480-482.
  \item Yang C. H. Lagrange Identity for Polinomials and $\delta$-Codes of Lengths 7t and 13t// Proc. Amer. Math. Soc. 1983. V. 88. № 4. P. 746-750.
  \item Агаян С. С., Саруханян А. Г. Обобщенные $\delta$-коды и построение матриц Адамара// Пробл. передачи информ. 1980. Т. 16. № 3. С. 50-59.
  \item Саруханян А. Г. О построении обобщенных последовательностей с нулевыми автокорреляционными функциями и матриц Адамара// Математические вопросы кибернетики и вычислительной техники. Ереван: Изд-во АН АрмССР, 1984. Т. 12. С. 105-129.
  \item Агаян С. С., Саруханян А. Г. Рекуррентные формулы построения матриц тиша Вильямсона // Мат. заметки, 1981. Т. 30. № 4. С. 603-617.
\end{enumerate}

Поступила в редакцию\\
2.IV. 1985

\section*{ЗАМЕЧАНИЕ О РЕШЕНИИ КВАДРАТНЫХ УРАВНЕНИЙ НАД ПОЛЯМИ ГАЛУА }
Предлагается формула для решения квадратных уравнений над полями Галуа $G F\left(2^{m}\right)$.

При декодировании алгебраических кодов возникает задача определения корней многочлена над полем Галуа. Значительный интерес представляет решение квадратных уравнений. В [1] получены формулы для


\end{document}